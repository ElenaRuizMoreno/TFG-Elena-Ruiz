\capitulo{6}{Conclusiones}

Desde que comencé este proyecto, he tenido muy presente que se trataba de una oportunidad real para contribuir, aunque fuera en una pequeña parte, a mejorar la calidad asistencial de recién nacidos prematuros en contextos vulnerables. El objetivo inicial era ambicioso: dotar a una incubadora de bajo coste de un sistema funcional de pulsioximetría, capaz de estimar en tiempo real la frecuencia cardíaca y la saturación de oxígeno. Sin embargo, con el paso de las semanas, ese objetivo se fue transformando en un reto que ha abarcado desde la fisiología neonatal y el procesamiento de señales biomédicas, hasta la programación embebida y la validación práctica sobre datos reales.

En este trabajo se ha investigado a fondo el funcionamiento interno de la pulsioximetría y su aplicabilidad al sistema que se planteaba. Se han analizado diferentes técnicas de filtrado y estimación de parámetros fisiológicos, implementándolas en Python y posteriormente adaptándolas al software original de la incubadora. También se ha tenido en cuenta la eficiencia computacional, la estabilidad de las señales y la necesidad de garantizar resultados estables a pesar del ruido que este tipo de sensores ocasiona. El diseño del sistema ha seguido una filosofía de código abierto y bajo coste, sin renunciar a la precisión de los resultados.

Como resultado, se ha conseguido una primera versión funcional del sistema, capaz de registrar y procesar señales PPG en tiempo real, generar una estimación de la frecuencia cardíaca y la SpO$_2$ que, si bien aún presentan limitaciones, sienta un punto de partida sobre el que seguir mejorando. 

Se ha desarrollado una técnica para almacenar y filtrar registros; se ha documentado todo el proceso de desarrollo en un repositorio y se ha validado experimentalmente la viabilidad del enfoque propuesto.

A lo largo de este proceso, también he tomado conciencia de la enorme diferencia que puede suponer una implementación bien pensada en un contexto de bajos recursos. Si algo me llevo de este trabajo, más allá del conocimiento técnico, es la convicción de que la ingeniería aplicada a la salud debe estar al servicio de las personas, especialmente cuando los medios son limitados. Porque diseñar un dispositivo para este tipo de contextos implica centrarse en lo esencial: que funcione, pueda reproducirse y responda a su finalidad.

\section{Aspectos relevantes.}

En líneas generales, se han podido cumplir los objetivos propuestos. Se
han conseguido crear los algoritmos capaces de traducir las señales PPG crudas de un sensor a valores fisiológicos legibles e interpretables por un humano. Sin embargo, a lo largo del desarrollo de este proyecto, han surgido múltiples decisiones técnicas, dificultades prácticas y aprendizajes que han influido directamente en el resultado final. 

Este apartado pretende recoger de los hitos más importantes, justificando los caminos tomados y resaltando aquellos aspectos que podrían resultar útiles para futuros desarrollos similares.

\begin{itemize}
\item \textbf{Inicio del proyecto y contextualización:} El punto de partida fue entender la realidad clínica y social que motiva el proyecto: mejorar el cuidado neonatal en países en vías de desarrollo mediante tecnologías de bajo coste. Esta necesidad guió todas las decisiones posteriores, tanto a nivel de hardware como de software, reforzando el enfoque práctico del sistema.
\item \textbf{Comprensión del entorno de desarrollo:} Uno de los primeros retos fue entender el funcionamiento del sistema de adquisición de señales, así como su configuración. Tanto el lenguaje de programación C++, como el uso de herramientas específicas como PlatformIO, Visual Studio Code o el uso de un programador que transfiriera los datos desde la PCB, eran prácticamente nuevos para mí. Esto supuso una fase inicial de adaptación, en la que fue necesario invertir tiempo en entender cómo funcionaba la arquitectura del sistema, cómo compilar el código y dónde modificar el código base para conseguir el objetivo final, sin apenas referencias previas.

Además, hay que destacar que, aunque el TFG se centra en el desarrollo del sistema de pulsioximetría, este componente está integrado dentro del sistema completo de la incubadora neonatal. Esto conllevó tener que comprender y trabajar sobre una base de código grande, cuyas modificaciones suponen un coste y tiempo computacional considerables, lo que añadió un nivel adicional de complejidad a cada cambio realizado. 

Igualmente, la primera adquisición y visualización de datos tardó más de la cuenta en suceder porque se dieron una serie de problemas técnicos que tenían que ser resueltos en un laboratorio con la ayuda del tutor. En cuanto al hardware, hubo que soldar los pines con los que el programador se conecta con la placa, ya que recibimos el material sin este ajuste realizado. Asimismo, fue necesario probar la ejecución del firmware con varios cables micro-USB porque los que disponíamos no eran reconocidos por el ordenador, y por último, la clavija del programador tuvo que ser apretada para que hiciera buen contacto. En cuanto al software, se tuvo que volver a desplegar el proyecto entero una segunda vez porque al ser tan grande, desde la ruta donde se había alojado inicialmente no se podía ejecutar.

Todos los detalles sobre la puesta en marcha del sistema completo, y las adaptaciones que se realizaron, serán detalladas en el \textit{anexo G}, en el \textit{Cuaderno de Trabajo}.

\item \textbf{Desarrollo del sistema de adquisición y almacenamiento de datos:} Al comienzo, no se sabía lo que el sistema iba a devolver, y al lograr visualizarlo, se observó que iba a ser complicado trabajar directamente en el entorno del firmware y que era necesario exportar estos datos para poder trabajar con ellos de manera independiente. 

Como ya se ha explicado, los datos consistían en la intensidad de las señales IR y RED de los LED del sensor, que si se graficaban frente al tiempo, dan como resultado una señal PPG. El primer inconveniente fue la falta de una referencia temporal de cada registro, y la frecuencia era demasiado elevada, lo que hacía imposible poder trabajar con los datos directamente. Se identificó la función específica del firmware responsable del envío de datos a la terminal con el formato observado y se diseñó un sistema que permitiera capturar datos crudos durante un periodo de 30 segundos, guardarlos en formato CSV e incluir una cabecera que aportara información sobre las señales registradas. 

Esta parte del proyecto facilitó el tratamiento de los datos sin necesidad de realizar modificaciones constantes en el firmware ni de mantener la placa conectada de forma continua.

\item \textbf{Validación experimental:} Se llevaron a cabo sesiones de adquisición de señales con los valores fisiológicos propios de la autora del proyecto en distintas situaciones. Una parte importante del trabajo fue clasificar registros válidos e inválidos, cortar tramos ruidosos y aplicar transformadas de Fourier para estudiar la componente espectral de las señales. Estos análisis permitieron afinar mejor los parámetros de los filtros y del algoritmo de estimación.

Cabe destacar que no se ha podido validar el sistema con recién nacidos prematuros ni en situaciones clínicas reales, por lo que se desconoce si el rendimiento observado en adultos sanos se mantendría en estos pacientes. La frecuencia cardíaca de los recién nacidos prematuros es significativamente más alta que la de un adulto, y en las sesiones de prueba no se pudo elevar la frecuencia cardíaca lo suficiente como para simular este escenario. Además, el sensor empleado es especialmente sensible al movimiento, lo que podría suponer un inconveniente en pacientes tan inestables.

\item \textbf{Análisis y filtrado de señales:} Durante el análisis de las señales PPG, se optó por intentar estimar por separado la frecuencia cardíaca y la saturación de oxígeno, ya que, tras haber revisado la documentación técnica y científica relacionada, estos cálculos eran muy distintos entre sí y se consideró necesario estudiarlos de manera individual. La mayoría de las veces, el método de trabajo era el mismo; se probaban los filtros y posteriores algoritmos sobre un único registro y, cuando se llegaba a un resultado fiable, se aplicaba a todos los demás para comprobar que no fuera azar.

\item \textbf{Estimación de HR y SpO$_2$:} Para la frecuencia cardíaca, como ha sido discutido en la metodología y posteriormente mostrado en los resultados, se probaron gran variedad de combinaciones de filtros y algoritmos, pero se optó por implementar un algoritmo basado en la detección de picos con filtros suaves de media+mediana, optimizado posteriormente para ejecución embebida. En el caso de SpO$_2$, no hubo tanta variedad de métodos porque todo se basa en una misma filosofía, pero se implementó el método \textit{Ratio of Ratios}, observando que sus resultados eran sensibles al ruido y a las variaciones de la señal. Se estudiaron alternativas como la regresión lineal, aunque no se integraron finalmente en el sistema.

\item \textbf{Despliegue embebido y documentación:} Se adaptaron todos los algoritmos al microcontrolador ESP32, cuidando el uso de memoria y el tiempo de ejecución. El proyecto se documentó completamente en un repositorio GitHub, incluyendo scripts de análisis, logs de prueba, y los anexos, donde se incluyen las instrucciones de instalación y manuales de usuario y programador. Esta documentación permite que el proyecto pueda ser replicado o ampliado en el futuro por otros desarrolladores o investigadores.

\item \textbf{Consideraciones sobre la diversidad de la población objetivo:} La incubadora \textit{In$^3$ator} está pensada para ser distribuida en países en vías de desarrollo, como algunos del sur de África o de Sudamérica. En estos lugares, la coloración de la piel de los recién nacidos puede afectar la absorción de luz por los tejidos y, por tanto, la fiabilidad del cálculo de SpO$_2$. Aunque el principio de funcionamiento de la pulsioximetría tiene en cuenta estas variaciones mediante el cálculo relativo de señales IR y RED, diversos estudios han demostrado que los dispositivos comerciales pueden infraestimar la saturación de oxígeno en personas con tonos de piel más oscuros.

\end{itemize}

En conjunto, este proyecto ha sido una experiencia formativa, que ha enriquecido mucho mi aprendizaje y mi experiencia como ingeniera en el ámbito de la salud. No solo se ha conseguido implementar un sistema de pulsioximetría medianamente funcional, sino que se han superado obstáculos técnicos relevantes que en el inicio no se consideraban y se ha creado una base sólida para su mejora futura.

