\capitulo{1}{Introducción}


Se estima que en 2020 nacieron de forma prematura 13,4 millones de bebés en todo el mundo. Uno de cada diez nacimientos. Pero detrás de esa cifra hay una desigualdad enorme: mientras que en países de altos ingresos la mayoría de estos recién nacidos sobrevive, en los países con menos recursos muchos no lo consiguen, sobre todo por la falta de cuidados básicos como una fuente de calor, oxígeno o sistemas de monitorización.

Más del 90\% de los bebés nacidos antes de las 28 semanas fallecen en los primeros días en los países con bajos ingresos, frente a menos del 10\% en países desarrollados. No es solo una cuestión médica; es también tecnológica y estructural. Mejorar el acceso a equipos sanitarios adecuados marca la diferencia entre vivir o no.

Conscientes de esta realidad, en 2014 un grupo de profesionales de la salud y la ingeniería puso en marcha el proyecto \textit{In$^3$ator}, una incubadora neonatal de bajo coste, accesible y de código abierto. Desde 2019, la ONG Medicina Abierta al Mundo lidera su fabricación y distribución en aquellos lugares que lo necesitan. Pero el proyecto sigue evolucionando. Una de sus necesidades actuales es integrar un sistema de monitorización fisiológica sencillo, fiable y adaptado a sus limitaciones técnicas.

Desde el inicio, el reto ha sido claro: convertir señales ópticas crudas en información médica fiable dentro de un sistema embebido y de bajo coste. A lo largo del proyecto se ha desarrollado un pipeline completo de tratamiento de señal, adaptando algoritmos a las limitaciones del hardware de la incubadora y validando su funcionamiento con datos reales.

Todo el contenido del proyecto, como el código, scripts, documentación y resultados, está disponible en el repositorio de GitHub TFG-Elena-Ruiz, organizado para facilitar su comprensión, reutilización e integración futura en el sistema \textit{In$^3$ator}.







