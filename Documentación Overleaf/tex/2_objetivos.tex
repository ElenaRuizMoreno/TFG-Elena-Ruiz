\capitulo{2}{Objetivos}

Este trabajo surge de una oportunidad muy especial, en la que se plantea la idea de integrar una herramienta adicional a un sistema médico de bajo coste, actualmente implantado con éxito alrededor del mundo. El objetivo principal es mejorar este dispositivo, desarrollando una funcionalidad complementaria que sea capaz de monitorizar a un neonato, midiendo en tiempo real sus constantes de saturación en sangre y frecuencia cardíaca. 


\section{Objetivos funcionales}
\begin{itemize}
    \item Desarrollar una primera versión funcional de un sistema que permita estimar, de forma sencilla, parámetros fisiológicos básicos a partir de señales ópticas.
    \item Comprobar, a través de pruebas controladas, si los resultados obtenidos permiten extraer información útil sobre el estado del paciente.
    \item  Garantizar que el trabajo realizado pueda servir como base para futuras mejoras o integraciones del dispositivo.
\end{itemize}

\section{Objetivos técnicos}
    \begin{itemize}
        \item Estudiar los fundamentos teóricos necesarios para procesar la señal fisiológica de interés.
        \item Implementar una solución que sea comprensible, reutilizable y compatible con los recursos de hardware disponibles.
        \item Investigar el mejor método de implementación del algoritmo, que se adapte a las limitaciones derivadas de los materiales utilizados. Priorizando la simplicidad, la estabilidad y la eficiencia.
    \end{itemize}

\section{Objetivos personales}
    \begin{itemize}
        \item Profundizar en el conocimiento del funcionamiento de la pulsioximetría y la fisiología neonatal.
        \item Adquirir mayor práctica en el tratamiento digital de señales biomédicas, incluyendo técnicas de filtrado, detección de artefactos y validación de resultados.
        \item Mejorar el manejo de herramientas de desarrollo embebido, entornos de análisis y gestión de proyectos con control de versiones.
        \item Habituación al uso del sistema de composición de texto LaTeX de cara a futuro para la redacción de documentos técnicos.
    \end{itemize}








